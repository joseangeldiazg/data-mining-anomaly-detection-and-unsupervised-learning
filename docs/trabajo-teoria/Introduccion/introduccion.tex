%---------------------------------------------------
% Nombre: capitulo1.tex  
% 
% Texto del capitulo 1
%---------------------------------------------------

\chapter{Introducci�n}
\label{introduccion}

Actualmente nadie deber�a sorprenderse cuando escuche que vivimos en la \textit {sociedad de la informaci�n}, concepto acu�ado para referenciar a una sociedad cambiante y donde la manipulaci�n de datos e informaci�n juega un papel m�s que relevante en las actividades sociales, culturales y sobre todo, econ�micas.  El tratamiento de estos datos puede suponer una ardua labor, m�s a�n cuando el volumen de estos es tan grande que los paradigmas para su procesado deben migrar hacia nuevas vertientes y a�n m�s cuando estos datos provienen de fuentes tan dispares como nuestras tendencias en la compra diaria, el uso que le damos a una tarjeta de cr�dito o a una red social... Es por ello, que fruto de la necesidad del an�lisis y la obtenci�n de informaci�n de estos datos en especie desestructurados y aparentemente carentes de significado surgen t�cnicas y herramientas capaces de procesar y obtener informaci�n �til y relevante. 

Como hemos mencionado anteriormente, las redes sociales son grandes factor�as de datos. Datos que una vez procesados pueden servir de ayuda para comprender temas relevantes de la sociedad actual o incluso desvelar patrones aparentemente ocultos en los h�bitos de comportamiento de usuarios que pueden ser de ayuda en procesos de toma de decisiones o  para diversos estudios posteriores. A este proceso se le denomina miner�a de redes sociales o \textit{social media mining} y es una de las vertientes de estudio sobre la que m�s se investiga actualmente dentro del �mbito de la miner�a de datos.

Continuaci�n, haremos una breve introducci�n al problema a resolver, para continuar con la puntualizaci�n de los objetivos principales del proyecto de fin de m�ster y concluir la secci�n dando al lector una idea de la organizaci�n final de la memoria. 

\section{Objetivos del trabajo}



\section{Organizaci�n del trabajo} 


\clearpage
%---------------------------------------------------