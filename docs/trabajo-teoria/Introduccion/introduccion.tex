%---------------------------------------------------
% Nombre: capitulo1.tex  
% 
% Texto del capitulo 1
%---------------------------------------------------

\chapter{Introducci�n}
\label{introduccion}

Actualmente nadie deber�a sorprenderse cuando escuche que vivimos en la \textit {sociedad de la informaci�n}, concepto acu�ado para referenciar a una sociedad cambiante y donde la manipulaci�n de datos e informaci�n juega un papel m�s que relevante en las actividades sociales, culturales y sobre todo, econ�micas.  El tratamiento de estos datos puede suponer una ardua labor, m�s a�n cuando el volumen de estos es tan grande que los paradigmas para su procesado deben migrar hacia nuevas vertientes y a�n m�s cuando estos datos provienen de fuentes tan dispares como nuestras tendencias en la compra diaria, el uso que le damos a una tarjeta de cr�dito o a una red social... Es por ello, que fruto de la necesidad del an�lisis y la obtenci�n de informaci�n de estos datos en especie des-estructurados y aparentemente carentes de significado surgen t�cnicas y herramientas capaces de procesar y obtener informaci�n �til y relevante. 

Una de estas t�cnicas es la miner�a de datos, que podr�a ser definida como el proceso de obtenci�n de informaci�n relevante y no tirivial sobre  conjuntos de datos, de manera que esta puede ser utilizada en los procesos de toma de decisiones de empresas o entidades, sin olvidar el papel acad�mico e investigador donde el uso de estas t�cnicas es innumerable. 

Dentro del �rea de la miner�a de datos, encontramos adem�s distintos enfoques. Estos enfoques pueden ir en funci�n de diversos factores, pero sin duda la divisi�n de t�cnicas de miner�a de datos m�s extendida, es la que las divide entre t�cnicas dirigidas o aprendizaje supervisado y t�cnicas no dirigidas, o aprendizaje no supervisado. 


\section{Organizaci�n del trabajo} 


\clearpage
%---------------------------------------------------